\documentclass[a4paper,12pt]{scrreprt}
\usepackage[T1]{fontenc}
\usepackage[utf8]{inputenc}
\usepackage[ngerman]{babel}
\usepackage[table]{xcolor}% http://ctan.org/pkg/xcolor
\usepackage{tabu}
\usepackage{graphicx}
\usepackage{lmodern}

\begin{document}


%\titlehead{Kopf} %Optionale Kopfzeile
\author{Backhausen Dominik \and Dimitrijevic Daniel} %Zwei Autoren
\title{ Loadbalancing } %Titel/Thema
\subject{VSDB} %Fach
%\subtitle{ } %Genaueres Thema, Optional
\date{\today} %Datum
\publishers{5AHITT} %Klasse

\maketitle
\tableofcontents


\chapter{Aufgabenstellung}
Aufgabenstellung

Es soll ein Load Balancer mit mindestens 2 unterschiedlichen Load-Balancing Methoden (jeweils 7 Punkte) implementiert werden (ähnlich dem PI Beispiel [1]; Lösung zum Teil veraltet [2]). Eine Kombination von mehreren Methoden ist möglich. Die Berechnung bzw. das Service ist frei wählbar!

Folgende Load Balancing Methoden stehen zur Auswahl:

    Least Connection
    Weighted Distribution
    Response Time
    Server Probes

Um die Komplexität zu steigern, soll zusätzlich eine "Session Persistence" (2 Punkte) implementiert werden.
Tests

Die Tests sollen so aufgebaut sein, dass in der Gruppe jedes Mitglied mehrere Server fahren und ein Gruppenmitglied mehrere Anfragen an den Load Balancer stellen.

Modalitäten

Gruppenarbeit: 2 Personen
Abgabe: Protokoll, Testszenarien, Sourcecode (mit allen notwendigen Bibliotheken), Java-Doc, Jar


Viel Erfolg!
	
\chapter{Designüberlegungen}
	
\chapter{Arbeitsaufteilung}
	\tabulinesep = 4pt
	\begin{tabu}  {|[2pt]X[2.5,c] |[1pt] X[4,c] |[1pt]X[1.3,c]|[1pt]X[c]|[2pt]}
		\tabucline[2pt]{-}
		Name & Arbeitssegment & Time Estimated & Time Spent\\\tabucline[2pt]{-}
		
		Backhausen Dominik & Verbindung & 1h & 1h\\\tabucline[1pt]{-}
		Backhausen Dominik & Implementierung der Loadbalancing Varianten & 2h & 4h\\\tabucline[1pt]{-}
		Daniel Dimitriejvic & Protokoll erstellung & 1h & 1h\\\tabucline[1pt]{-}
		Dimitrijevic Daniel & Verbindung& 1h & 1h\\\tabucline[2pt]{-}
		Gesamt && 105h & 15h\\\tabucline[2pt]{-}
	\end{tabu}	
\chapter{Arbeitsdurchführung}
\chapter{Testbericht}
\chapter{Quellen}
[1] "Praktische Arbeit 2 zur Vorlesung 'Verteilte Systeme' ETH Zürich, SS 2002", Prof.Dr.B.Plattner, übernommen von Prof.Dr.F.Mattern (http://www.tik.ee.ethz.ch/tik/education/lectures/VS/SS02/Praktikum/aufgabe2.pdf)
[2] http://www.tik.ee.ethz.ch/education/lectures/VS/SS02/Praktikum/loesung2.zip

\end{document}